\documentclass[11pt, a4paper]{article}

\usepackage[utf8]{inputenc}
\usepackage{fullpage}
\usepackage{graphicx}
\graphicspath{ {images/} }
\usepackage{tabularx}
\usepackage{listings}
\usepackage{mathtools}
\usepackage{amssymb}
\usepackage{eurosym}
\usepackage[ngerman]{babel}
\usepackage{cancel} %Kürzen

\providecommand{\abs}[1]{\left\lvert#1\right\rvert}
\providecommand{\norm}[1]{\left\lVert#1\right\rVert}
\providecommand{\acos}{\mathrm{acos}}
\providecommand{\asin}{\mathrm{asin}}
\providecommand{\dx}{\ \mathrm{dx}}
\providecommand{\dt}{\ \mathrm{dt}}

\title{Klausurvorbereitung Statistik \\ Aufgaben}
\author{Yannick Schrör \and Christian Mielers}
\date{\today}

\begin{document}
\maketitle
\section{Wahrscheinlichkeiten}
Bei der Auswertung der Statistikklausuren aus den letzten 4 Jahren haben sich für folgende Themenbereiche folgende Häufigkeiten ergeben:\\

\begin{tabular}{r|l|r|r}
	ID 	&	Thema 										& Häufigkeit 	& Wahrscheinlichkeit\\ \hline
	1	& Anwendungen von Verteilungen 					& 7 			& \\
	2	& Stichproben 									& 8				& \\
	3	& Dichte- und Verteilungsfunktionen 			& 7 			& \\
	4	& Wahrscheinlichkeitstabelle / Abhängigkeiten 	& 4 			& \\
	5	& Urnenmodell / Kombinatorik 					& 4 			& \\
	6	& Erwartungswert und Varianz 					& 2 			& \\
\end{tabular}\\

\begin{enumerate}
\item Gehen Sie davon aus, dass Sie zu Übungszwecken zufällig eine Aufgabe aus den Altklausuren aussuchen und bearbeiten möchten. Berechnen Sie für jedes einzelne Thema die Wahrscheinlichkeit, dass Sie eine Aufgabe seines Typs auswählen.
\item Nehmen Sie an, eine Klausur bestünde aus zwei Aufgaben. Der nachfolgenden Tabelle können Sie entnehmen, mit welcher Wahrscheinlichkeit ein Thema in einer Altklausur drankam, wenn eines der anderen Thema ebenfalls drankam. Vervollständigen Sie die Tabelle.


\begin{figure}[h!]
\centering
\setlength\extrarowheight{5pt}
\begin{tabular}{r|cccccc|r}
	  		& 1 	& 2 	& 3 	& 4 	& 5 	& 6 	& $\sum$\\ \hline
	1 		& 0 	& 5/32 	&  		& 0  	& 0  	& 0 	& 7/32 \\
	2 		& 5/32	& 0		& 1/32 	& 0 	&   	&   	& 8/32 \\
	3 		& 		& 1/32 	& 0 	& 3/32 	& 1/32 	& 0  	& 7/32 \\
	4 		& 0		& 0 	& 3/32 	& 0 	& 0  	&   	& 4/32 \\
	5 		& 0  	&   	& 1/32 	& 0 	& 0  	& 1/32 	& 4/32 \\
	6 		& 0 	&   	& 0  	&  		& 1/32 	& 0 	& 2/32 \\ \hline
	$\sum$ 	& 7/32 	& 8/32 	& 7/32 	& 4/32 	& 4/32 	& 2/33  & 1\\
\end{tabular}
\end{figure}

\item Ist das Auftreten der Aufgaben statistisch unabhängig?
\item Bestimmen Sie folgende Wahrscheinlichkeiten:

	\begin{tabular}{ll}
	a) $P(5)$ \hspace{5cm} 				& c) $P(1|3)$\\
	b) $P(3 \land 5)$ 					& d) $P((2 \land 5) \lor (3 \land 4))$
	\end{tabular}

\end{enumerate}

\newpage
\section{Urnenmodell}
Die Übung zur Vorlesung Statistik wird von 42 Leuten besucht, von denen 35 die Aufgaben vorbereitet haben, die restlichen 7 sind unvorbereitet. Es werden zufällig 3 Stundenten ausgewählt, die die Aufgaben vorrechnen sollen. Wie groß ist die Wahrscheinlichkeit, dass
\begin{enumerate}
	\item keiner der Studenten auffliegt
	\item genau zwei der Studenten auffliegen
\end{enumerate}
wenn wir von Ziehen mit (ohne) Zurücklegen mit (ohne) Beachtung der Reihenfolge ausgehen?

\vspace{\baselineskip}
\begin{tabular}{|c||c|c|}
	\hline
	\multicolumn{3}{|c|}{Aus Menge mit \textbf{n} Elementen \textbf{k} ziehen} \\
	\hline
	 & \textbf{mit} Beachtung der Reihenfolge & \textbf{ohne} Beachtung der Reihenfolge \\
	\hline \hline
	\textbf{mit} Zurücklegen & $n^k$ & $\dbinom{n+k-1}{k} = \dfrac{(n+k-1)!}{k!\cdot (n-1)!}$ \\
	\hline
	\textbf{ohne} Zurücklegen & $\dfrac{n!}{(n-k)!}$ & $\dbinom{n}{k} = \dfrac{n!}{k!\cdot (n-k)!}$ \\
\hline
\end{tabular}


\newpage
\section{Dichte- und Verteilungsfunktion}
Yannick

\newpage
\section{Anwendung von Verteilungen}
Chrisbot

\newpage
\section{Stichproben}

\end{document}









