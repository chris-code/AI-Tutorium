\documentclass[11pt, a4paper]{article}

\usepackage[utf8]{inputenc}
\usepackage{fullpage}
\usepackage{graphicx}
\graphicspath{ {images/} }
\usepackage{tabularx}
\usepackage{listings}
\usepackage{enumitem}
\usepackage{mathtools}
\usepackage{amssymb}
\usepackage{eurosym}
\usepackage[ngerman]{babel}
\usepackage{cancel} %Kürzen
%\usepackage{comment}
%\usepackage{color}
%\usepackage{xcolor}
\usepackage{tikz}
%\usepackage{tkz-euclide}

\definecolor{DarkGreen}{rgb}{0,0.53,0}

%TODO
%Hinweisen auf: Körperaufgabe, Statistik-Material zur Kombinatorik

\providecommand{\abs}[1]{\left\lvert#1\right\rvert}
\providecommand{\norm}[1]{\left\lVert#1\right\rVert}
\providecommand{\acos}{\mathrm{acos}}
\providecommand{\asin}{\mathrm{asin}}
\providecommand{\dx}{\ \mathrm{dx}}
\providecommand{\dt}{\ \mathrm{dt}}
\providecommand\br[1]{\left(#1\right)}
\providecommand\ubr[2]{\underbrace{#1}_{#2}}
\providecommand\setequal{\overset{!}{=}}

\allowdisplaybreaks

\title{Klausurvorbereitung Höhere Mathematik \\ Lösungen}
\author{Yannick Schrör \and Christian Mielers}
\date{5. März 2015}

\begin{document}
\maketitle

\section{Ungleichungen}
Bestimme die Lösungen für folgende Ungleichungen
\begin{enumerate}
	\item $\displaystyle \frac{\abs{x+2}}{2} > \frac{x}{3} + 1$
		\begin{align*}
			\frac{\abs{x+2}}{2} > \frac{x}{3} + 1 &\Leftrightarrow \abs{x+2} > \frac{2x}{3} + 2 \\
			&\Leftrightarrow \abs{x+2} > \frac{2x+6}{3}
			\intertext{Fall 1: $x \in \left( -\infty, -2 \right)$}
			-(x+2) > \frac{2x + 6}{3} &\Leftrightarrow -x -2 > \frac{2x + 6}{3} \\
			&\Leftrightarrow -3x -6 > 2x + 6 \\
			&\Leftrightarrow -12 > 5x \\
			&\Leftrightarrow \frac{-12}{5} > x
			\intertext{Lösung im Fall 1: $\left( -\infty, -2 \right) \cap \left( -\infty, \frac{-12}{5} \right) = \left( -\infty, \frac{-12}{5} \right)$}
			\intertext{Fall 2: $x \in \left[ -2, \infty \right)$}
			(x+2) > \frac{2x + 6}{3} &\Leftrightarrow 3x + 6 > 2x + 6 \\
			&\Leftrightarrow x > 0
			\intertext{Lösung im Fall 3: $\left[ -2, \infty \right) \cap \left( 0, \infty \right) = \left( 0, \infty \right)$}
		\end{align*}
	\item $\displaystyle \sqrt{x^2 - 4x + 4} < \frac{\abs{x+1}}{2}$
		\begin{align*}
			\sqrt{x^2 - 4x + 4} < \frac{\abs{x+1}}{2} &\Leftrightarrow \sqrt{(x-2)^2} < \frac{\abs{x+1}}{2} \\
			&\Leftrightarrow \abs{x-2} < \frac{\abs{x+1}}{2}
			\intertext{Fall 1: $x \in \left( -\infty, -1 \right)$}
			-(x-2) < \frac{-(x+1)}{2} &\Leftrightarrow x-2 > \frac{x+1}{2} \\
			&\Leftrightarrow 2x-4 > x+1 \\
			&\Leftrightarrow x > 5
			\intertext{Lösung im Fall 1: $\left( -\infty, -1 \right) \cap \left( 5, \infty \right) = \emptyset$}
			\intertext{Fall 2: $x \in \left[ -1, 2 \right)$}
			-(x-2) < \frac{(x+1)}{2} &\Leftrightarrow -x+2 < \frac{x+1}{2} \\
			&\Leftrightarrow -2x+4 < x+1 \\
			&\Leftrightarrow 3 < 3x \\
			&\Leftrightarrow 1 < x
			\intertext{Lösung im Fall 2: $\left[ -1, 2 \right) \cap \left( 1, \infty \right) = \left( 1,2 \right)$}
			\intertext{Fall 3: $x \in \left[ 2, \infty \right)$}
			(x-2) < \frac{(x+1)}{2} &\Leftrightarrow 2x-4 < x+1 \\
			&\Leftrightarrow x < 5
			\intertext{Lösung im Fall 3: $\left[ 2, \infty \right) \cap \left( -\infty, 5 \right) = \left[ 2,5 \right)$}
		\end{align*}
		Damit ist die Gesammtlösung: $\emptyset \cup \left( 1,2 \right) \cup \left[ 2,5 \right) = \left( 1, 5 \right)$
	\item $\displaystyle \abs{x+5} > 4 + \sqrt{x^2 + 6x + 9}$
		\begin{align*}
			\abs{x+5} > 4 + \sqrt{x^2 + 6x + 9} &\Leftrightarrow \abs{x+5} > 4 + \sqrt{(x+3)^2} \\
			&\Leftrightarrow \abs{x+5} > 4 + \abs{x+3} \\
			&\Leftrightarrow \abs{x+5} - \abs{x+3} > 4
		\end{align*}
		$\abs{x+5}$ and $\abs{x+3}$ sind maximal 2 voneinander entfernt, daher gibt es keine Lösung.
\end{enumerate}
\newpage
\section{Vollständige Induktion}
\begin{enumerate}[label=\alph*)]
	\item Zeige mittels vollständiger Induktion, dass die folgende Gleichung für alle $n \in \mathbb{N}$ gilt: \\
		$\displaystyle \sum_{i=1}^n 4i - 1 = 2n^2 + n$
		\begin{align*}
			\intertext{\textbf{Lösung:}}
			\intertext{\textbf{Induktionsanfang:} Wir zeigen, dass die Gleichung für $n = 1$ gilt:}
			\sum_{i=1}^1 4i - 1 &= 2 \cdot 1^2 + 1\\
			4 \cdot 1 - 1 &= 2 + 1\\
			3 &= 3 \qquad \quad \textcolor{DarkGreen}{\checkmark}
			\intertext{\textbf{Induktionsschritt:} Nun müssen wir zeigen, dass die Gleichung auch für $n + 1$ gilt:}
			\sum_{i=1}^{n+1} 4i - 1 &= 2(n+1)^2 + (n+1)\\
			\intertext{Die Summe bis $n+1$ können wir in die Summe bis $n$ und den $(n+1)$ten Summanden aufteilen:}
			\br{\sum_{i=1}^n 4i - 1} + \br{4(n+1) - 1} &= 2(n+1)^2 + (n+1)\\
			\intertext{Nach der Induktionsannahme (Aufgabenstellung) können wir die Summe bis $n$ wie folgt ersetzen:}
			(2n^2 + n) + (4(n+1) -1) &= 2(n+1)^2 + (n+1)\\
			\intertext{Nun erhalten wir durch Vereinfachen beider Seiten:}
			2n^2 + n + 4n + 4 - 1 &= 2(n^2 + 2n + 1) + n + 1\\
			2n^2 + 5n + 3 &= 2n^2 + 4n + 2 + n + 1\\
			2n^2 + 5n + 3 &= 2n^2 + 5n + 3 \qquad \quad \textcolor{DarkGreen}{\checkmark}
		\end{align*}
	\newpage
	\item Zeige mittels vollständiger Induktion, dass folgende Gleichung für alle $n \in \mathbb{N}$ gilt: \\
		$\displaystyle \sum_{i=1}^n i \cdot (i + 1) = \frac{n\cdot(n+1)\cdot(n+2)}{3}$
		\begin{align*}
			\intertext{\textbf{Lösung:}}
			\intertext{\textbf{Induktionsanfang:} Wir zeigen, dass die Gleichung für $n = 1$ gilt:}
			\sum_{i=1}^1 i \cdot (i + 1) &= \frac{1 \cdot (1 + 1) \cdot (1 + 2)}{3}\\
			1 \cdot 2 &= \frac{1 \cdot 2 \cdot \cancel{3}}{\cancel{3}}\\
			2 &= 2 \qquad \quad \textcolor{DarkGreen}{\checkmark}
			\intertext{\textbf{Induktionsschritt:} Nun müssen wir zeigen, dass die Gleichung auch für $n + 1$ gilt:}
			\sum_{i=1}^{n+1} i \cdot (i + 1) &= \frac{(n + 1)\cdot((n+1) + 1)\cdot((n+1)+2)}{3}\\
			\sum_{i=1}^{n+1} i \cdot (i + 1) &= \frac{(n + 1)\cdot(n + 2)\cdot(n + 3)}{3}\\
			\intertext{Die Summe bis $n+1$ können wir in die Summe bis $n$ und den $(n+1)$ten Summanden aufteilen:}
			\br{\sum_{i=1}^{n} i \cdot (i + 1)} + (n+1) \cdot (n+2) &= \frac{(n + 1)\cdot(n + 2)\cdot(n + 3)}{3}\\
			\intertext{Nach der Induktionsannahme (Aufgabenstellung) können wir die Summe bis $n$ wie folgt ersetzen:}
			\frac{n \cdot (n + 1) \cdot (n + 2)}{3} + (n+1) \cdot (n+2) &= \frac{(n + 1)\cdot(n + 2)\cdot(n + 3)}{3}\\
			\intertext{Durch Multiplikation mit Faktor 3 erhalten wir:}
			%\frac{n \cdot (n + 1) \cdot (n + 2)}{3} + \frac{3 \cdot (n+1) \cdot (n+2)}{3} &= \frac{(n + 1)\cdot(n + 2)\cdot(n + 3)}{3}\\
			n \cdot (n + 1) \cdot (n + 2) + 3 \cdot (n+1) \cdot (n+2) &= (n + 1)\cdot(n + 2)\cdot(n + 3)\\
			\intertext{Da $n \geq 1$ können wir nun auf beiden Seiten durch $(n+1) \cdot (n+2)$ dividieren:}
			n + 3 &= n + 3 \qquad \quad \textcolor{DarkGreen}{\checkmark}
		\end{align*}
\end{enumerate}

\newpage
\section{Komplexe Zahlen}
	\begin{description}
		\item[Betrag] $\abs{x+yi} = \sqrt{x^2 + y^2}$
		\item[Polarkoordinatenform] $x+yi = r \left( \cos{\alpha} + i \sin{\alpha} \right) \text{mit } r=\abs{x+yi}, \alpha = \acos{\frac{x}{r}} = \asin{\frac{y}{r}}$
		\item[Polarkoordinaten potenzieren] $\left( r \left( \cos{\alpha} + i \sin{\alpha} \right) \right)^{p} = r^p \left( \cos \left( p \alpha \right) + i \sin \left(p \alpha \right) \right)$
		\item[Polarkoordinaten n-te Wurzel ziehen] $w_k = \sqrt[n]{r} \left( \cos \frac{\alpha + 2k\pi}{n} + i \sin \frac{\alpha + 2k\pi}{n} \right)$ \\ für $k = 0 \dots n-1$
	\end{description}
	\begin{enumerate}
		\item Bestimme den Realteil, Imaginärteil, Betrag und die komplex Konjugierte von $\displaystyle \frac{2+i}{1-i}$
			\begin{align*}
				\frac{2+i}{1-i} &= \frac{(2+i) \cdot (\overline{1-i})}{(1-i) \cdot (\overline{1-i})} \\
				&= \frac{(2+i) \cdot (1+i)}{(1-i) \cdot (1+i)} \\
				&= \frac{2 + 3i + i^2}{1 - i^2} \\
				&= \frac{1 + 3i}{2} \\
				&= \frac{1}{2} + \frac{3}{2} i
			\end{align*}
			Realteil: $\frac{1}{2}$, Imaginärteil: $\frac{3}{2}$ \\
			Betrag: $\abs{\frac{1}{2} + \frac{3}{2} i} = \sqrt{(\frac{1}{2})^2 + (\frac{3}{2})^2} = \sqrt{\frac{1}{4} + \frac{9}{4}} = \sqrt{\frac{5}{2}}$ \\
			Komplex Konjugierte: $\overline{\frac{1}{2} + \frac{3}{2} i} = \frac{1}{2} - \frac{3}{2} i$
		\item Berechne $\displaystyle (1-i)^{32}$
			\begin{align*}
				\intertext{Zunächst $1-i$ in Polarkoordinaten umrechnen.}
				r &= \abs{1-i} = \sqrt{1^2 + (-1)^2} = \sqrt{2} \\
				\alpha &= \acos{\frac{1}{\sqrt{2}}} = \asin{\frac{-1}{\sqrt{2}}} = \frac{7\pi}{4}
				\intertext{Es gilt also: $1-i = \sqrt{2} \left( \cos{\frac{7\pi}{4} + i \sin{\frac{7\pi}{4}}} \right)$. Potenzieren ist mit Polarkoordinaten einfach:}
				\left( \sqrt{2} \left( \cos{\frac{7\pi}{4} + i \sin{\frac{7\pi}{4}}} \right) \right)^{32} &= \sqrt{2}^{32} \left( \cos \left( 32 \cdot \frac{7\pi}{4} \right) + i \sin \left( 32 \cdot \frac{7\pi}{4} \right) \right) \\
				&= 2^{16} \left( \cos \left( 56\pi \right) + i \sin \left( 56\pi \right) \right) \\
				&= 2^{16} \left( \cos \left( 0 \right) + i \sin \left( 0 \right) \right) \\
				&= 2^{16} \left( 1 + 0i \right) \\
				&= 2^{16}
			\end{align*}
		\item Für welche $\displaystyle p\in\mathbb{C}$ gilt $p^4 + 4 = 0$?
			\begin{align*}
				p^4 + 4 = 0 &\Leftrightarrow p^4 = -4 \\
				&\Leftrightarrow p = \sqrt[4]{-4}
				\intertext{Um die Wurzel zu ziehen, wandeln wir $-4$ in Polarkoordinaten um}
				r &= 4 \\
				\alpha &= \acos{\frac{-4}{4}} = \asin \frac{0}{4} = \pi
				\intertext{Nun berechnen wir}
				\sqrt[4]{4 \left( \cos \pi + i \sin \pi \right)} &\overset{k=0}{=} \sqrt[4]{4} \left( \cos \frac{\pi + 0\pi}{4} + i \sin \frac{\pi + 0\pi}{4} \right) = \sqrt{2} \left( \frac{\sqrt{2}}{2} + i \frac{\sqrt{2}}{2} \right) = 1+i \\
				&\overset{k=1}{=} \sqrt[4]{4} \left( \cos \frac{\pi + 2\pi}{4} + i \sin \frac{\pi + 2\pi}{4} \right) = \sqrt{2} \left( -\frac{\sqrt{2}}{2} + i \frac{\sqrt{2}}{2} \right) = -1+i \\
				\intertext{(An dieser Stelle könnten wir aufhören, die anderen beiden Lösungen sind die komplex konjugierten der bereits berechneten)}
				&\overset{k=2}{=} \sqrt[4]{4} \left( \cos \frac{\pi + 4\pi}{4} + i \sin \frac{\pi + 4\pi}{4} \right) = \sqrt{2} \left( -\frac{\sqrt{2}}{2} - i \frac{\sqrt{2}}{2} \right) = -1-i \\
				&\overset{k=3}{=} \sqrt[4]{4} \left( \cos \frac{\pi + 6\pi}{4} + i \sin \frac{\pi + 6\pi}{4} \right) = \sqrt{2} \left( \frac{\sqrt{2}}{2} - i \frac{\sqrt{2}}{2} \right) = 1-i 
			\end{align*}
	\end{enumerate}

\newpage
\section{Folgen}
\begin{description}
	\item[Folgen 1. Ordnung] $a_{n+1} = q \cdot a_n + d$
		\begin{align*}
			a_n = \begin{cases}
				a_0 \cdot q^n + d \cdot \frac{1-q^n}{1-q}, &\text{ falls } q \in \mathbb{R} \setminus \{1\} \\
				a_0 + d \cdot n &\text{ falls } q=1
					\end{cases}
		\end{align*}
	\item[Folgen 2. Ordnung] $a_{n+1} = b a_n + c a_{n-1}$ \vspace{0.3cm} \\
		Charakteristisches Polynom bilden: $q^2 = bq+c$ \vspace{0.3cm} \\
		\begin{tabular}{|c|l|}
			\hline
			Diskriminante & Explizite Darstellung \\ \hline
			$>0$ & $a_n = C_1 q_1^n + C_2 q_2^n$ \\
			$=0$ & $a_n = C_1 q_1^n + C_2 n \cdot q_2^n$ \\
			$<0$ & $a_n = C_1 r^n \cos(n\alpha) + C_2 r^n \sin(n\alpha)$ \\ \hline
		\end{tabular}
\end{description}
Finde eine explizite Darstellung für $a_n$. Gib auch den Grenzwert $\displaystyle\lim_{n \mapsto \infty} a_n$ an.
\begin{enumerate}
	\item $2 \cdot a_{n+3} = 10 \cdot a_{n+2} + 16, \qquad a_1 = 28$
		\begin{align*}
			\intertext{Zunächst ist die Gleichung in die Form $a_{n+1} = q \cdot a_n + d$ zu bringen}
			2 \cdot a_{n+3} = 10 \cdot a_{n+2} + 16 &\Leftrightarrow a_{n+3} = 5 \cdot a_{n+2} + 8 \tag{:2} \\
			&\Leftrightarrow a_{n+1} = 5 \cdot a_{n} + 8 \tag{Index-Shift}
			\intertext{Nun ist $a_0$ auszurechnen. Dazu kann in diesem Fall $n=0$ gesetzt werden.}
			a_1 &= 5 \cdot a_0 + 8 \\
			\Leftrightarrow 28 &= 5 \cdot a_0 + 8 \\
			\Leftrightarrow 4 &= a_0
			\intertext{Die Gleichung hat also die Form $a_{n+1} = 5 \cdot a_{n} + 8, \qquad a_0 = 4$. Damit gehen wir in die Lösungsgleichung} \\
			a_n &= a_0 \cdot q^n + d \cdot \frac{1-q^n}{1-q} \tag{da $q \neq 1$} \\
			&= 4 \cdot 5^n + 8 \cdot \frac{1-5^n}{1-5} \\
			&= 4 \cdot 5^n - 2 \cdot \left( 1-5^n \right) \\
			&= 4 \cdot 5^n - 2 + 2 \cdot 5^n \\
			&= 6 \cdot 5^n - 2
		\end{align*}
		Es gilt: $\lim_{n \mapsto \infty} 6 \cdot 5^n - 2 = +\infty$
	\item $a_{n+1} = 0 \cdot a_n - 4 \cdot a_{n-1}, \qquad a_0 = 1, a_1 = 3$
		\begin{align*}
			\intertext{Zunächst bilden wir das charakteristische Polynom mit $b=0, c=-4$}
			q^2 &= 0q - 4 \\
			q_{1,2} &= \sqrt{-4} \\
			\intertext{Die Diskriminante ist also $< 0$. Wir bestimmen die komplexen Lösungen und bestimmen r und $\alpha$ der Polarkoordinaten.}
			q_{1,2} &= \pm 2i \\
			r &= \sqrt{0^2 + 2^2} = 2 \\
			\alpha &= \acos \left( \frac{0}{r} \right) = \asin \left( \frac{2}{r} \right) = \frac{\pi}{2}
			\intertext{Diese Werte für r und $\alpha$ setzen wir in die Formel ein}
			a_n &= C_1 2^n \cos \left(n \frac{\pi}{2} \right) + C_2 2^n \sin \left( n \frac{\pi}{2} \right)
			\intertext{Abschließend bestimmen wir $C_1$ und $C_2$ aus den Anfangsbedingungen}
			&\begin{cases}
				a_0 = C_1 r^0 \cos \left( 0 \cdot \frac{\pi}{2} \right) + C_2 r^0 \sin \left( 0 \cdot \frac{\pi}{2} \right) \\
				a_1 = C_1 r^1 \cos \left( 1 \cdot \frac{\pi}{2} \right) + C_2 r^1 \sin \left( 1 \cdot \frac{\pi}{2} \right)
			\end{cases} \tag{Variablen einsetzen} \\
			&\begin{cases}
				1 = C_1 \cos \left( 0 \right) + C_2 \sin \left( 0 \right) \\
				3 = C_1 \cdot 2 \cdot \cos \left( \frac{\pi}{2} \right) + C_2 \cdot 2 \cdot \sin \left( \frac{\pi}{2} \right)
			\end{cases} \tag{Manche Terme ergeben sich zu 0} \\
			&\begin{cases}
				1 = C_1 \\
				\frac{3}{2} = C_2
			\end{cases}
			\intertext{Da wir nun $C_1$ und $C_2$ kennen, können wir die explizite Form angeben}
			a_n &= 2^n \cos \left( n \frac{\pi}{2} \right) + \frac{3}{2} \cdot 2^n \sin \left( n \frac{\pi}{2} \right)
		\end{align*}
		Es gilt: $\lim_{n \mapsto \infty} 2^n \cos \left( n \frac{\pi}{2} \right) + \frac{3}{2} \cdot 2^n \sin \left( n \frac{\pi}{2} \right) = \tilde{\infty}$ (alternierend divergent)
\end{enumerate}

\newpage
\section{Reihen}
Sind die folgenden Reihen konvergent oder divergent?
\begin{enumerate}
	\item $\displaystyle \sum_{k=0}^\infty \frac{(k-1)(k+3)}{2(k-2)(k-4)}$
		\begin{align*}
			\sum_{k=0}^\infty \frac{(k-1)(k+3)}{2(k-2)(k-4)} &= \sum_{k=0}^\infty \frac{k^2 + 2k - 3}{2(k^2 - 6k + 8)} \\
			&= \sum_{k=0}^\infty \frac{k^2 + 2k - 3}{2k^2 - 12k + 16}
		\end{align*}
		Die Reihe ist divergent, da $\displaystyle{\lim_{k \mapsto \infty} a_k = \frac{1}{2} \neq 0}$, wobei $a_k$ die Reihenglieder sind.
	\item $\displaystyle \sum_{k=0}^\infty \frac{(-1)^k 2^{k-1} - 3 \cdot 5^k}{3^{2k+1}}$
		\begin{align*}
			\intertext{Wir konzentrieren uns zunächst auf die Reihenglieder}
			a_k = \frac{(-1)^k 2^{k-1} - 3 \cdot 5^k}{3^{2k+1}} &= \frac{(-1)^k \frac{2^k}{2} - \frac{3\cdot2}{2} \cdot 5^k}{3^{2k} \cdot 3} \\
			&= \frac{(-1)^k 2^k - 6 \cdot 5^k}{3^{2k} \cdot 6} \\
			&= \frac{(-2)^k - 6 \cdot 5^k}{9^k \cdot 6} \\
			&= \frac{(-2)^k}{9^k \cdot 6} - \frac{\cancel{6} \cdot 5^k}{9^k \cdot \cancel{6}} \\
			&= \frac{1}{6} \cdot \left(\frac{-2}{9}\right)^k - \left(\frac{5}{9}\right)^k
			\intertext{Zurück zur Reihe}
			\sum_{k=0}^\infty \frac{(-1)^k 2^{k-1} - 6 \cdot 5^k}{3^{2k+1}} &= \sum_{k=0}^\infty \frac{1}{6} \cdot \left(\frac{-2}{9}\right)^k - \left(\frac{5}{9}\right)^k \\
			&= \frac{1}{6} \underbrace{\sum_{k=0}^\infty \left(\frac{-2}{9}\right)^k}_{\text{geom. Reihe}} - \underbrace{\sum_{k=0}^\infty \left(\frac{5}{9}\right)^k}_{\text{geom. Reihe}}
			\intertext{Für die geometrische Reihe gilt $\sum_{k=0}^\infty q^k = \frac{1}{1-q}$ bei $\abs{q} < 1$}
			= \frac{1}{6} \sum_{k=0}^\infty \left(\frac{-2}{9}\right)^k - \sum_{k=0}^\infty \left(\frac{5}{9}\right)^k &= \frac{1}{6} \cdot \frac{1}{1 - \frac{-2}{9}} - \frac{1}{1 - \frac{5}{9}} \\
			&= \frac{1}{6} \cdot \frac{1}{\frac{11}{9}} - \frac{1}{\frac{4}{9}} \\
			&= - \frac{93}{44}
		\end{align*}
\end{enumerate}

\newpage
\section{Differentialrechnung}
Bestimme den maximalen Definitionsbereich in $\mathbb{R}$, die Nullstellen, die Grenzwerte an den Randpunkten des Definitionsbereichs, die lokalen und globalen Extrema sowie die Monotonie folgender Funktion:
\begin{align*}
	f(x) &= (x+4) \cdot (2-x^2)^{-\frac{1}{2}} - 3
\end{align*}

\begin{align*}
	\intertext{\textbf{Lösung:}}
	\intertext{Zur Bestimmung des maximalen Definitionsbereichs vereinfachen wir die Funktion zunächst:}
	&= (x+4) \cdot \frac{1}{(2-x^2)^\frac{1}{2}} - 3\\
	&= (x+4) \cdot \frac{1}{\sqrt{2 - x^2}} - 3
	\intertext{Zunächst betrachten wir $\sqrt{2-x^2}$. Die Wurzel darf nicht negativ werden, da wir sonst komplexe Zahlen erhalten. Somit können wir den maximalen Definitionsbereich zunächst auf $D = [-\sqrt{2},\sqrt{2}]$ einschränken. Hierbei lassen wir jedoch $0$ als Ergebnis der Wurzel zu. Da die Wurzel jedoch im Nenner steht, müssen wir $0$ ausschließen. Somit erhalten wir $D = (-\sqrt{2},\sqrt{2})$.}
	 \intertext{Als nächstes bestimmen wir die Nullstellen durch Gleichsetung von $f(x)$ mit $0$:}
	 0 &\setequal f(x)\\
	 0 &= (x+4) \cdot \frac{1}{\sqrt{2 - x^2}} - 3\\
	 3 &= (x+4) \cdot \frac{1}{\sqrt{2 - x^2}}\\
	 3\sqrt{2 - x^2} &= x + 4 \tag{Nun quadrieren wir!}\\
	 9 (2 - x^2) &= (x + 4)^2\\
	 18 - 9x^2 &= x^2 + 8x + 16\\
	 0 &= 10x^2 + 8x -2
	 \intertext{Zur Ermittlung der Nullstellen wenden wir die Mitternachtsformel an:}
	 x_{1,2} &= \frac{-b \pm \sqrt{b^2 - 4ac}}{2a}\\
	 &= \frac{-8 \pm \sqrt{8^2 - 4 \cdot 10 \cdot (-2)}}{2 \cdot 10}\\
	 &= \frac{-8 \pm \sqrt{64 + 80}}{20}\\
	 &= \frac{-8 \pm \sqrt{144}}{20}\\
	 &= \frac{-8 \pm 12}{20}\\
	 x_1 &= -1 \qquad \qquad x_2 = \frac{1}{5}\\
	 \intertext{Beide Werte liegen innerhalb des Definitionsbereichs. Durch Einsetzen sieht man ein, dass beide Werte gültige Lösungen sind:}
	 f(x_1) = f(-1) &= (-1+4) \cdot (2-(-1)^2)^{-\frac{1}{2}} - 3\\
	 &= 3 \cdot 1 - 3\\
	 &= 0\\
	 f(x_1) = f\br{\frac{1}{5}} &= \br{\frac{1}{5} + 4} \cdot \br{2-\br{\frac{1}{5}}^2}^{-\frac{1}{2}} - 3\\
	 &= \frac{21}{5} \cdot \br{\frac{49}{25}}^{-\frac{1}{2}} - 3\\
	 &= \frac{21}{5} \cdot \sqrt{\frac{25}{49}} - 3 \\
	 &= \frac{21}{5} \cdot \frac{5}{7} - 3 \\
	 &= 0
	 \\
	 \intertext{Als nächstes berechnen wir die Grenzwerte an den Randpunkten des Definitionsbereichs}
	 \lim_{x \rightarrow {-\sqrt{2}}_+} f(x) &= \lim_{x \rightarrow {-\sqrt{2}}_+} (x+4) \cdot \frac{1}{\sqrt{2 - x^2}} - 3\\
	 &= (-\sqrt{2} + 4) \cdot \frac{1}{0_+} - 3\\
	 &= +\infty - 3\\
	 &= +\infty\\
	 \lim_{x \rightarrow {\sqrt{2}}_-} f(x) &= \lim_{x \rightarrow {\sqrt{2}}_-} (x+4) \cdot \frac{1}{\sqrt{2 - x^2}} - 3\\
	 &= (\sqrt{2} + 4) \cdot \frac{1}{0_+} - 3\\
	 &= +\infty - 3\\
	 &= +\infty
	 \intertext{Als nächstes berechnen wir die lokalen und globalen Extrema. Dazu berechnen wir die Ableitung von $f(x)$ und setzen sie gleich $0$:}
	 \frac{\text{d}}{\dx} f(x) &= \frac{\text{d}}{\dx} \br{(x+4) \cdot (2-x^2)^{-\frac{1}{2}} - 3}\\
	 &= \frac{\text{d}}{\dx} \br{(x+4) \cdot (2-x^2)^{-\frac{1}{2}}}\\
	 \intertext{Wir wenden nun die Produktregel $(f(x)\cdot g(x))' = f'(x)g(x) + f(x)g'(x)$ an:} %TODO hart kürzen
	 &= 1 \cdot (2-x^2)^{-\frac{1}{2}} + (x+4) \cdot \br{-\frac{1}{2}} \cdot (2-x^2)^{-\frac{3}{2}} \cdot (-2x)\\
	 &= (2-x^2)^{-\frac{1}{2}} + (x^2+4x) \cdot \br{(2-x^2)^{-\frac{1}{2}}}^3\\
	 &= (2-x^2)^{-\frac{1}{2}} + (x^2+4x) \cdot \br{(2-x^2)^{-\frac{1}{2}}}^2 \cdot (2-x^2)^{-\frac{1}{2}}\\
	 &= (2-x^2)^{-\frac{1}{2}} \cdot \br{1 + (x^2+4x) \cdot \br{(2-x^2)^{-\frac{1}{2}}}^2}\\
	 &= (2-x^2)^{-\frac{1}{2}} \cdot \br{1 + (x^2+4x) \cdot \frac{1}{2-x^2}}\\
	 &= \frac{1}{\sqrt{2-x^2}} \cdot \br{1 + \frac{x^2 + 4x}{2-x^2}}\\
	 &= \frac{1}{\sqrt{2-x^2}} \cdot \br{\frac{4x + 2}{2-x^2}}\\
	 \intertext{Nun setzen wir die Ableitung gleich 0:}
	 0 &\setequal \frac{1}{\sqrt{2-x^2}} \cdot \br{\frac{4x + 2}{2-x^2}}\\
	 \intertext{Wir stellen fest, dass der linke Faktor nicht 0 werden kann. Somit erhalten wir:}
	 0 &= \frac{4x + 2}{2-x^2}
	 \intertext{Setzt man einen Bruch gleich 0, muss nur der Zähler betrachtet werden:}
	 0 &= 4x + 2\\
	 x &= -\frac{1}{2}
	 \intertext{Nun kennen wir also alle Kandidaten für lokale oder globale Extrema. Da das Berechnen der zweiten Ableitung von $f(x)$ recht aufwändig wäre, betrachten wir den Globalverlauf von $f(x)$.}
\end{align*}
\vspace{-1.7cm}
\begin{figure}[h!]
\centering
\begin{tabular}{c|ccccc} %TODO wie ist die f'-Zeile zu lesen?
	$x$ & $-\sqrt{2}$ &  & $-\frac{1}{2}$ &   & $\sqrt{2}$\\ \hline
	$f(x)$ & $+\infty$ & $\searrow$ & $\approx -2,0646$ & $\nearrow$ & $+\infty$\\
	$f'(x)$ & & $-$ & 0 & $+$ & \\
\end{tabular}
\end{figure}\\
Wir stellen fest, dass am linken Rand von $D$ $f(x)$ sehr groß ist und monoton fällt, bis wir bei $x = -\frac{1}{2}$ das globale Minimum erreicht haben. Anschließend steigt die Funktion monoton bis zum rechten Rand von $D$.

\begin{figure}[h!]
	\centering
	\begin{tikzpicture}[yscale=0.19,xscale=3.0]
		\draw[->] (-2,0) -- (2,0) node[right] {$x_1$};
		\draw[shift={(1.41,0)},color=black] (0pt,3pt) -- (0pt,-3pt) node[below] {$\sqrt{2}$};
		\draw[shift={(-1.41,0)},color=black] (0pt,3pt) -- (0pt,-3pt) node[below] {$-\sqrt{2}$};
		\draw[shift={(-0.5,0)},color=black] (0pt,3pt) -- (0pt,-3pt) node[below] {$-\frac{1}{2}$};
		\draw[shift={(-1,0)},color=black] (0pt,3pt) -- (0pt,-3pt) node[below] {$-1$};
		\draw[shift={(0.2,0)},color=black] (0pt,3pt) -- (0pt,-3pt) node[below] {$\frac{1}{5}$};
		\draw[shift={(-0.5,0)},color=blue] (0pt,40pt) -- (0pt,-40pt) node[above] {};
		\node at (-0.5,3) [draw=none,color=blue] {Minimum};
		\draw[->] (0,-4) --(0,28) node[above] {$f(x)$};
		\draw[domain=-1.412:1.405,samples=1000,variable=\x,red] plot ({\x},{((\x + 4) * (1/(sqrt(2 - (\x * \x)))) - 3});
	\end{tikzpicture}
\end{figure}

\newpage
\section{Partialbruchzerlegung und Substitution}
\begin{align}
	\int_a^b \mathrm{f}( \ \underbrace{\mathrm{g}(x)}_t \ ) \dx = \int_{\mathrm{g}(a)} ^{\mathrm{g(b)}} \mathrm{f}(t) \dt
	\label{eq:substitution}
\end{align}

Berechne $\displaystyle \int_2^3 \frac{6 \sin(x) \cos(x)}{\sin^2(x) + \sin(x) - 2 } dx$

\begin{align*}
	\intertext{Zunächst substituieren wir $\sin(x)$.}
	t &= \sin(x) \\
	\frac{\dt}{\dx} &= \cos(x) \Leftrightarrow \dx = \frac{\dt}{\cos(x)} \\
	\int_2^3 \frac{6 \sin(x) \cos(x)}{\sin^2(x) + \sin(x) - 2 } \dx &= \int_{\sin(2)}^{\sin(3)} \frac{6 \ t \ \cancel{\cos(x)}}{t^2 + t - 2} \ \frac{\dt}{\cancel{\cos(x)}} \\
	&= \int_{\sin(2)}^{\sin(3)} \frac{6 \ t}{t^2 + t - 2} \dt \tag{PQ-Formel} \\
	&= \int_{\sin(2)}^{\sin(3)} \frac{6 \ t}{(t-1)(t+2)} \dt \\
	\intertext{Nun wenden wir den Ansatz der PBZ auf den Integranden an}
	\frac{6 \ t}{(t-1)(t+2)} &= \frac{A}{t-1} + \frac{B}{t+2} \\
	&= \frac{A (t+2)}{(t-1)(t+2)} + \frac{B (t-1)}{(t-1)(t+2)} \\
	&= \frac{A (t+2) + B (t-1)}{(t-1)(t+2)} \tag{Zähler gleichsetzen} \\
	6t &= A (t+2) + B (t-1) \\
	\intertext{Wir setzen nun die Nullstellen für t ein, um Lösungen für A und B zu erhalten}
	6 \cdot 1 &= A (1+2) + B (1-1) \Leftrightarrow A = 2 \tag{t=1} \\
	6 \cdot (-2) &= A ((-2)+2) + B ((-2)-1) \Leftrightarrow B = 4 \tag{t=-2} \\
	\intertext{Jetzt wo wir A und B kennen, wissen wir aus dem Ansatz der PBZ}
	\frac{6 \ t}{(t-1)(t+2)} &= \frac{2}{t-1} + \frac{4}{t+2} \\
	\intertext{Diese Gleichung wenden wir auf das Integral an}
	 \int_{\sin(2)}^{\sin(3)} \frac{6 \ t}{(t-1)(t+2)} \dt &=  \int_{\sin(2)}^{\sin(3)} \frac{2}{t-1} + \frac{4}{t+2} \dt \\
	 &= \int_{\sin(2)}^{\sin(3)} \frac{2}{t-1} \dt + \int_{\sin(2)}^{\sin(3)} \frac{4}{t+2} \dt \\
	 &= 2 \int_{\sin(2)}^{\sin(3)} \frac{1}{t-1} \dt + 4 \int_{\sin(2)}^{\sin(3)} \frac{1}{t+2} \dt \\
	 &= 2 \left. \left( \ln\abs{t-1} \right) \right\vert_{\sin(2)}^{\sin(3)} + 4 \left. \left( \ln\abs{t+2} \right) \right\vert_{\sin(2)}^{\sin(3)} \\
	 &= 2 \left( \ln\abs{\sin(3)-1} - \ln\abs{\sin(2)-1} \right) + 4 \left( \ln\abs{\sin(3) + 2} - \ln\abs{\sin(2) + 2} \right) \\
	 %&= 2 \left( \ln{0.8589} \right) \\
	 &= 2 \left( \ln \frac{\abs{\sin(3)-1}}{\abs{\sin(2)-1}} \right) + 4 \left( \ln \frac{\abs{\sin(3) + 2}}{\abs{\sin(2) + 2}} \right) \\
	 &= 2 \left( \ln \abs{\frac{\sin(3)-1}{\sin(2)-1}} \right) + 4 \left( \ln \abs{\frac{\sin(3) + 2}{\sin(2) + 2}} \right) \\
	 &\approx 3.2698
\end{align*}

\newpage
\section{Partielle Integration}
\begin{align}
	\int_a^b f'(x) \cdot g(x) \dx =& f(x) \cdot g(x)\vert_a^b - \int_a^b f(x) \cdot g'(x) \dx
	\label{eq:partielleIntegration}
\end{align}

\newcommand{\lowerBorder}{{0}}
\newcommand{\upperBorder}{{\frac{1}{3}}}
Löse folgendes Integral: $\displaystyle \int_\lowerBorder^\upperBorder e^{3x} \cdot (2x - 5) \dx$

\begin{align*}
	\intertext{\textbf{Lösung:}}
	\intertext{Wähle $f'$ und $g$ wie folgt:}
	& \int_\lowerBorder^\upperBorder \ubr{e^{3x}}{f'} \cdot \ubr{\br{2x - 5}}{g}\dx\\
	\intertext{Um die Formel anwenden zu können, benötigen wir die Stammfunktion $f$ von $f'$ und die Ableitung $g'$ von $g$.}
	f =&\int f' \dx\\
	=&\int e^{3x} \dx\\
	=&\ \frac{1}{3} e^{3x}\\
	g' =& \frac{\partial}{\partial x} (2x - 5)\\
	=& 2\\
	\intertext{In Formel \eqref{eq:partielleIntegration} müssen wir nur noch die entsprechenden Ersetzungen machen:}
	&f(x) \cdot g(x)\vert_a^b - \int_a^b f(x) \cdot g'(x) \dx\\
	=& \br{\frac{1}{3} e^{3x} \cdot \br{2x - 5}}\Big\vert_\lowerBorder^\upperBorder - \int_\lowerBorder^\upperBorder \frac{1}{3} e^{3x} \cdot 2 \dx\\
	=& \br{\br{\frac{1}{3} e^1 \cdot \br{2\cdot \frac{1}{3} - 5}} - \br{\frac{1}{3}e^0 \cdot \br{0 - 5}}} - \frac{2}{3} \int_\lowerBorder^\upperBorder e^{3x} \dx\\
	=& \br{\br{\frac{1}{3}e \cdot \br{-\frac{13}{3}}} - \br{-\frac{5}{3}}} - \frac{2}{3}\br{\frac{1}{3}e^{3x}\Big\vert_\lowerBorder^\upperBorder}\\
	=& \br{-\frac{13}{9}e + \frac{5}{3}} - \frac{2}{9} \br{e^{3x}\Big\vert_\lowerBorder^\upperBorder}\\
	=& -\frac{13}{9}e + \frac{5}{3} - \frac{2}{9} \br{e^1 - e^0}\\
	=& -\frac{13}{9}e + \frac{5}{3} - \frac{2}{9} \br{e - 1}\\
	=& -\frac{13}{9}e + \frac{15}{9} - \frac{2}{9}e + \frac{2}{9}\\
	=& -\frac{15}{9}e + \frac{17}{9}\\
	=& \frac{17}{9} - \frac{5}{3}e
	%%%%%%%%%%%%%%%%%%%%%%%%%%%%%%%%%%%%%%%%%%%%%%%%%%
	\intertext{\textbf{Alternative Lösung:}}
	& \int_\lowerBorder^\upperBorder e^{3x} \cdot (2x - 5) \dx\\
	=& \int_\lowerBorder^\upperBorder  2xe^{3x} - 5e^{3x} \dx\\
	=& \int_\lowerBorder^\upperBorder  2xe^{3x} \dx - \int_\lowerBorder^\upperBorder 5e^{3x} \dx\\
	=& 2\int_\lowerBorder^\upperBorder  xe^{3x} \dx - 5\int_\lowerBorder^\upperBorder e^{3x} \dx
	\intertext{Lösen wir zunächst das hintere Integral:}
	=& 2\int_\lowerBorder^\upperBorder xe^{3x} \dx - 5 \br{\frac{1}{3}e^{3x}\Big\vert_\lowerBorder^\upperBorder}\\
	=& 2\int_\lowerBorder^\upperBorder xe^{3x} \dx - 5 \br{\frac{1}{3}e^{3\frac{1}{3}} - \frac{1}{3}e^0}\\
	=& 2\int_\lowerBorder^\upperBorder xe^{3x} \dx - 5 \br{\frac{1}{3}e - \frac{1}{3}}\\
	\intertext{Kümmern wir uns nun um das vordere Integral. Hier multiplizieren wir ein Polynom mit einer Exponentialfunktion. In diesem Falle bietet sich die partielle Integration an. Um die Berechnungen möglichst kurz zu halten, betrachten wir das hintere -- bereits gelöste -- Integral nicht weiter, behalten es aber im Hinterkopf.}
	\intertext{Die Formel für die partielle Integration lautet:}
	\int_a^b f'(x) \cdot g(x) \dx =& f(x) \cdot g(x)\vert_a^b - \int_a^b f(x) \cdot g'(x) \dx
	\intertext{Wähle $f' = e^{3x}$ und $g = x$:}
	& \int_\lowerBorder^\upperBorder \ubr{x}{g} \cdot \ubr{e^{3x}}{f'}\dx
	\intertext{Um die Formel anwenden zu können, benötigen wir die Stammfunktion $f$ von $f'$ und die Ableitung $g'$ von $g$.}
	f =&\int f' \dx\\
	=&\int e^{3x} \dx\\
	=&\ \frac{1}{3} e^{3x}\\
	g' =&\ \frac{\partial}{\partial x}\ g\\
	=&\ 1
	\intertext{Da wir nun $f, f', g$ und $g'$ kennen, müssen wir die Komponenten nur noch in die Formel einsetzen. Dabei ignorieren wir zunächst den Vorfaktor 2. Diesen müssen wir später jedoch wieder berücksichtigen. Wir erhalten:}
	& \int_\lowerBorder^\upperBorder \ubr{x}{g} \cdot \ubr{e^{3x}}{f'} \dx\\
	=& \br{\frac{1}{3}e^{3x} \cdot x}\Big\vert_\lowerBorder^\upperBorder - \int_\lowerBorder^\upperBorder \frac{1}{3}e^{3x} \cdot 1 \dx\\
	=& \br{\br{\frac{1}{3}e^{3 \frac{1}{3}} \cdot \frac{1}{3}} - \br{\frac{1}{3}e^0 \cdot 0}} - \frac{1}{3} \int_\lowerBorder^\upperBorder e^{3x} \dx\\
	=& \frac{1}{9} e - \frac{1}{3}\br{\frac{1}{3}e^{3x}}\Big\vert_\lowerBorder^\upperBorder\\
	=& \frac{1}{9} e - \frac{1}{3}\br{\frac{1}{3}e^{3\frac{1}{3}} - \frac{1}{3}e^0}\\
	=& \frac{1}{9} e - \frac{1}{3}\br{\frac{1}{3}e - \frac{1}{3}}\\
	=& \frac{1}{9} e - \frac{1}{9}e + \frac{1}{9}\\
	=& \frac{1}{9}
	\intertext{Vergessen wir nicht die Multiplikation mit 2!}
	=& \frac{2}{9}
	\intertext{Ingesamt erhalten wir also:}
	=& \frac{2}{9} - 5 \br{\frac{1}{3}e - \frac{1}{3}}\\
	=& \frac{2}{9} - \frac{5}{3}e + \frac{5}{3}\\
	=& \frac{2}{9} - \frac{5}{3}e  + \frac{15}{9}\\
	=& \frac{17}{9} - \frac{5}{3}e\\
	\approx& -2.6416\\
\end{align*}

\end{document}









