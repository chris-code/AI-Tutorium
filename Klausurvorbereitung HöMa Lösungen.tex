\documentclass[11pt, a4paper]{article}

\usepackage[utf8]{inputenc}
\usepackage{fullpage}
\usepackage{graphicx}
\graphicspath{ {images/} }
\usepackage{tabularx}
\usepackage{listings}
\usepackage{enumitem}
\usepackage{mathtools}
\usepackage{amssymb}
\usepackage{eurosym}
\usepackage[ngerman]{babel}
\usepackage{cancel} %Kürzen
\usepackage{comment}
\usepackage{color}
\usepackage{xcolor}

\definecolor{DarkGreen}{rgb}{0,0.53,0}

%TODO
%Hinweisen auf: Körperaufgabe, Statistik-Material zur Kombinatorik

\providecommand{\abs}[1]{\left\lvert#1\right\rvert}
\providecommand{\norm}[1]{\left\lVert#1\right\rVert}
\providecommand{\acos}{\mathrm{acos}}
\providecommand{\asin}{\mathrm{asin}}
\providecommand{\dx}{\ \mathrm{dx}}
\providecommand{\dt}{\ \mathrm{dt}}
\providecommand\br[1]{\left(#1\right)}
\providecommand\ubr[2]{\underbrace{#1}_{#2}}

\title{Klausurvorbereitung Statistik \\ Aufgaben}
\author{Yannick Schrör \and Christian Mielers}
\date{5. März 2015}

\begin{document}
\maketitle

\section{Ungleichungen}
Bestimme die Lösungen für folgende Ungleichungen
\begin{enumerate}
	\item $\frac{\abs{x+2}}{2} > \frac{x}{3} + 1$
		\begin{align*}
			\frac{\abs{x+2}}{2} > \frac{x}{3} + 1 &\Leftrightarrow \abs{x+2} > \frac{2x}{3} + 2 \\
			&\Leftrightarrow \abs{x+2} > \frac{2x+6}{3}
			\intertext{Fall 1: $x \in \left( -\infty, -2 \right)$}
			-(x+2) > \frac{2x + 6}{3} &\Leftrightarrow -x -2 > \frac{2x + 6}{3} \\
			&\Leftrightarrow -3x -6 > 2x + 6 \\
			&\Leftrightarrow -12 > 5x \\
			&\Leftrightarrow \frac{-12}{5} > x
			\intertext{Lösung im Fall 1: $\left( -\infty, -2 \right) \cap \left( -\infty, \frac{-12}{5} \right) = \left( -\infty, \frac{-12}{5} \right)$}
			\intertext{Fall 2: $x \in \left[ -2, \infty \right)$}
			(x+2) > \frac{2x + 6}{3} &\Leftrightarrow 3x + 6 > 2x + 6 \\
			&\Leftrightarrow x > 0
			\intertext{Lösung im Fall 3: $\left[ -2, \infty \right) \cap \left( 0, \infty \right) = \left( 0, \infty \right)$}
		\end{align*}
	\item $\sqrt{x^2 - 4x + 4} < \frac{\abs{x+1}}{2}$
		\begin{align*}
			\sqrt{x^2 - 4x + 4} < \frac{\abs{x+1}}{2} &\Leftrightarrow \sqrt{(x-2)^2} < \frac{\abs{x+1}}{2} \\
			&\Leftrightarrow \abs{x-2} < \frac{\abs{x+1}}{2}
			\intertext{Fall 1: $x \in \left( -\infty, -1 \right)$}
			-(x-2) < \frac{-(x+1)}{2} &\Leftrightarrow x-2 > \frac{x+1}{2} \\
			&\Leftrightarrow 2x-4 > x+1 \\
			&\Leftrightarrow x > 5
			\intertext{Lösung im Fall 1: $\left( -\infty, -1 \right) \cap \left( 5, \infty \right) = \emptyset$}
			\intertext{Fall 2: $x \in \left[ -1, 2 \right)$}
			-(x-2) < \frac{(x+1)}{2} &\Leftrightarrow -x+2 < \frac{x+1}{2} \\
			&\Leftrightarrow -2x+4 < x+1 \\
			&\Leftrightarrow 3 < 3x \\
			&\Leftrightarrow 1 < x
			\intertext{Lösung im Fall 2: $\left[ -1, 2 \right) \cap \left( 1, \infty \right) = \left( 1,2 \right)$}
			\intertext{Fall 3: $x \in \left[ 2, \infty \right)$}
			(x-2) < \frac{(x+1)}{2} &\Leftrightarrow 2x-4 < x+1 \\
			&\Leftrightarrow x < 5
			\intertext{Lösung im Fall 3: $\left[ 2, \infty \right) \cap \left( -\infty, 5 \right) = \left[ 2,5 \right)$}
		\end{align*}
		Damit ist die Gesammtlösung: $\emptyset \cup \left( 1,2 \right) \cup \left[ 2,5 \right) = \left( 1, 5 \right)$
	\item $\abs{x+5} > 4 + \sqrt{x^2 + 6x + 9}$
		\begin{align*}
			\abs{x+5} > 4 + \sqrt{x^2 + 6x + 9} &\Leftrightarrow \abs{x+5} > 4 + \sqrt{(x+3)^2} \\
			&\Leftrightarrow \abs{x+5} > 4 + \abs{x+3} \\
			&\Leftrightarrow \abs{x+5} - \abs{x+3} > 4
		\end{align*}
		$\abs{x+5}$ and $\abs{x+3}$ sind maximal 2 voneinander entfernt, daher gibt es keine Lösung.
\end{enumerate}
\newpage
\section{Vollständige Induktion}
\begin{enumerate}[label=\alph*)]
\item
	Zeige mittels vollständiger Induktion, dass die folgende Gleichung für alle $n \in \mathbb{N}$ gilt:
	\begin{align*}
		\sum_{i=1}^n 4i - 1 &= 2n^2 + n \tag{Induktionsannahme}\\
		\intertext{\textbf{Lösung:}}
		\intertext{\textbf{Induktionsanfang:} Wir zeigen, dass die Gleichung für $n = 1$ gilt:}
		\sum_{i=1}^1 4i - 1 &= 2 \cdot 1^2 + 1\\
		4 \cdot 1 - 1 &= 2 + 1\\
		3 &= 3 \qquad \quad \textcolor{DarkGreen}{\checkmark}
		\intertext{\textbf{Induktionsschritt:} Nun müssen wir zeigen, dass die Gleichung auch für $n + 1$ gilt:}
		\sum_{i=1}^{n+1} 4i - 1 &= 2(n+1)^2 + (n+1)\\
		\intertext{Die Summe bis $n+1$ können wir in die Summe bis $n$ und den $(n+1)$ten Summanden aufteilen:}
		\br{\sum_{i=1}^n 4i - 1} + \br{4(n+1) - 1} &= 2(n+1)^2 + (n+1)\\
		\intertext{Nach der Induktionsannahme können wir die Summe bis $n$ wie folgt ersetzen:}
		(2n^2 + n) + (4(n+1) -1) &= 2(n+1)^2 + (n+1)\\
		\intertext{Nun erhalten wir durch Vereinfachen beider Seiten:}
		2n^2 + n + 4n + 4 - 1 &= 2(n^2 + 2n + 1) + n + 1\\
		2n^2 + 5n + 3 &= 2n^2 + 4n + 2 + n + 1\\
		2n^2 + 5n + 3 &= 2n^2 + 5n + 3 \qquad \quad \textcolor{DarkGreen}{\checkmark}
	\end{align*}
	\newpage
	\item
	Zeige mittels vollständiger Induktion, dass folgende Gleichung für alle $n \in \mathbb{N}$ gilt:
	\begin{align*}
		\sum_{i=1}^n i \cdot (i + 1) &= \frac{n\cdot(n+1)\cdot(n+2)}{3} \tag{Induktionsannahme}
		\intertext{\textbf{Lösung:}}
		\intertext{\textbf{Induktionsanfang:} Wir zeigen, dass die Gleichung für $n = 1$ gilt:}
		\sum_{i=1}^1 i \cdot (i + 1) &= \frac{1 \cdot (1 + 1) \cdot (1 + 2)}{3}\\
		1 \cdot 2 &= \frac{1 \cdot 2 \cdot \cancel{3}}{\cancel{3}}\\
		2 &= 2 \qquad \quad \textcolor{DarkGreen}{\checkmark}
		\intertext{\textbf{Induktionsschritt:} Nun müssen wir zeigen, dass die Gleichung auch für $n + 1$ gilt:}
		\sum_{i=1}^{n+1} i \cdot (i + 1) &= \frac{(n + 1)\cdot((n+1) + 1)\cdot((n+1)+2)}{3}\\
		\sum_{i=1}^{n+1} i \cdot (i + 1) &= \frac{(n + 1)\cdot(n + 2)\cdot(n + 3)}{3}\\
		\intertext{Die Summe bis $n+1$ können wir in die Summe bis $n$ und den $(n+1)$ten Summanden aufteilen:}
		\br{\sum_{i=1}^{n} i \cdot (i + 1)} + (n+1) \cdot (n+2) &= \frac{(n + 1)\cdot(n + 2)\cdot(n + 3)}{3}\\
		\intertext{Nach der Induktionsannahme können wir die Summe bis $n$ wie folgt ersetzen:}
		\frac{n \cdot (n + 1) \cdot (n + 2)}{3} + (n+1) \cdot (n+2) &= \frac{(n + 1)\cdot(n + 2)\cdot(n + 3)}{3}\\
		\intertext{Durch Vereinfachen erhalten wir:}
		\frac{n \cdot (n + 1) \cdot (n + 2)}{3} + \frac{3 \cdot (n+1) \cdot (n+2)}{3} &= \frac{(n + 1)\cdot(n + 2)\cdot(n + 3)}{3}\\
		n \cdot (n + 1) \cdot (n + 2) + 3 \cdot (n+1) \cdot (n+2) &= (n + 1)\cdot(n + 2)\cdot(n + 3)\\
		\intertext{Da $n \geq 1$ können wir nun auf beiden Seiten durch $(n+1) \cdot (n+2)$ dividieren:}
		n + 3 &= n + 3 \qquad \quad \textcolor{DarkGreen}{\checkmark}
	\end{align*}
\end{enumerate}
\newpage
\section{Komplexe Zahlen}
\subsection{Wurzelziehen}
Chrisbot
\section{Folgen}
Chrisbot
\section{Reihen}
Yannick
\section{Differentialrechnung}
Yannick
\section{Partialbruchzerlegung und Substitution}
Chrisbot

\newpage
\section{Partielle Integration}

Löse folgendes Integral:

\newcommand{\lowerBorder}{{0}}
\newcommand{\upperBorder}{{\frac{1}{3}}}

\begin{align*}
	& \int_\lowerBorder^\upperBorder e^{3x} \cdot (2x - 5) \dx\\
	\intertext{\textbf{Lösung:}}
	=& \int_\lowerBorder^\upperBorder  2xe^{3x} - 5e^{3x} \dx\\
	=& \int_\lowerBorder^\upperBorder  2xe^{3x} \dx - \int_\lowerBorder^\upperBorder 5e^{3x} \dx\\
	=& 2\int_\lowerBorder^\upperBorder  xe^{3x} \dx - 5\int_\lowerBorder^\upperBorder e^{3x} \dx
	\intertext{Lösen wir zunächst das hintere Integral:}
	=& 2\int_\lowerBorder^\upperBorder xe^{3x} \dx - 5 \br{\frac{1}{3}e^{3x}\Big\vert_\lowerBorder^\upperBorder}\\
	=& 2\int_\lowerBorder^\upperBorder xe^{3x} \dx - 5 \br{\frac{1}{3}e^{3\frac{1}{3}} - \frac{1}{3}e^0}\\
	=& 2\int_\lowerBorder^\upperBorder xe^{3x} \dx - 5 \br{\frac{1}{3}e - \frac{1}{3}}\\
	\intertext{Kümmern wir uns nun um das vordere Integral. Hier multiplizieren wir ein Polynom mit einer Exponentialfunktion. In diesem Falle bietet sich die partielle Integration an. Um die Berechnungen möglichst kurz zu halten, betrachten wir das hintere -- bereits gelöste -- Integral nicht weiter, behalten es aber im Hinterkopf.}
	\intertext{Die Formel für die partielle Integration lautet:}
	\int_a^b f'(x) \cdot g(x) \dx =& f(x) \cdot g(x)\vert_a^b - \int_a^b f(x) \cdot g'(x) \dx
	\intertext{Wähle $f' = e^{3x}$ und $g = x$:}
	& \int_\lowerBorder^\upperBorder \ubr{x}{g} \cdot \ubr{e^{3x}}{f'}\dx
	\intertext{Um die Formel anwenden zu können, benötigen wir die Stammfunktion $f$ von $f'$ und die Ableitung $g'$ von $g$.}
	f =&\int f' \dx\\
	=&\int e^{3x} \dx\\
	=&\ \frac{1}{3} e^{3x}\\
	g' =&\ \frac{\partial}{\partial x}\ g\\
	=&\ 1
	\intertext{Da wir nun $f, f', g$ und $g'$ kennen, müssen wir die Komponenten nur noch in die Formel einsetzen. Dabei ignorieren wir zunächst den Vorfaktor 2. Diesen müssen wir später jedoch wieder berücksichtigen. Wir erhalten:}
	& \int_\lowerBorder^\upperBorder \ubr{x}{g} \cdot \ubr{e^{3x}}{f'} \dx\\
	=& \br{\frac{1}{3}e^{3x} \cdot x}\Big\vert_\lowerBorder^\upperBorder - \int_\lowerBorder^\upperBorder \frac{1}{3}e^{3x} \cdot 1 \dx\\
	=& \br{\br{\frac{1}{3}e^{3 \frac{1}{3}} \cdot \frac{1}{3}} - \br{\frac{1}{3}e^0 \cdot 0}} - \frac{1}{3} \int_\lowerBorder^\upperBorder e^{3x} \dx\\
	=& \frac{1}{9} e - \frac{1}{3}\br{\frac{1}{3}e^{3x}}\Big\vert_\lowerBorder^\upperBorder\\
	=& \frac{1}{9} e - \frac{1}{3}\br{\frac{1}{3}e^{3\frac{1}{3}} - \frac{1}{3}e^0}\\
	=& \frac{1}{9} e - \frac{1}{3}\br{\frac{1}{3}e - \frac{1}{3}}\\
	=& \frac{1}{9} e - \frac{1}{9}e + \frac{1}{9}\\
	=& \frac{1}{9}
	\intertext{Vergessen wir nicht die Multiplikation mit 2!}
	=& \frac{2}{9}
	\intertext{Ingesamt erhalten wir also:}
	=& \frac{2}{9} - 5 \br{\frac{1}{3}e - \frac{1}{3}}\\
	=& \frac{2}{9} - \frac{5}{3}e + \frac{5}{3}\\
	=& \frac{2}{9} - \frac{5}{3}e  + \frac{15}{9}\\
	=& \frac{17}{9} - \frac{5}{3}e\\
	\approx& -2.6416\\
	\intertext{\textbf{Alternative Lösung:} Anstatt das Integral in zwei Integrale aufzuteilen, können wir auch direkt das ursprüngliche Integral mittels partieller Integration lösen.}
	& \int_\lowerBorder^\upperBorder \ubr{e^{3x}}{f'} \cdot \ubr{\br{2x - 5}}{g}\dx\\
	\intertext{Nun müssen wir wieder $f$ und $g'$ berechnen:}
	f =& \frac{1}{3}e^{3x} \tag{siehe oben}\\
	g' =& \frac{\partial}{\partial x} (2x - 5)\\
	=& 2\\
	\intertext{In der folgenden Formel müssen wir nur noch die entsprechenden Ersetzungen machen:}
	&f(x) \cdot g(x)\vert_a^b - \int_a^b f(x) \cdot g'(x) \dx\\
	=& \br{\frac{1}{3} e^{3x} \cdot \br{2x - 5}}\Big\vert_\lowerBorder^\upperBorder - \int_\lowerBorder^\upperBorder \frac{1}{3} e^{3x} \cdot 2 \dx\\
	=& \br{\br{\frac{1}{3} e^1 \cdot \br{2\cdot \frac{1}{3} - 5}} - \br{\frac{1}{3}e^0 \cdot \br{0 - 5}}} - \frac{2}{3} \int_\lowerBorder^\upperBorder e^{3x} \dx\\
	=& \br{\br{\frac{1}{3}e \cdot \br{-\frac{13}{3}}} - \br{-\frac{5}{3}}} - \frac{2}{3}\br{\frac{1}{3}e^{3x}\Big\vert_\lowerBorder^\upperBorder}\\
	=& \br{-\frac{13}{9}e + \frac{5}{3}} - \frac{2}{9} \br{e^{3x}\Big\vert_\lowerBorder^\upperBorder}\\
	=& -\frac{13}{9}e + \frac{5}{3} - \frac{2}{9} \br{e^1 - e^0}\\
	=& -\frac{13}{9}e + \frac{5}{3} - \frac{2}{9} \br{e - 1}\\
	=& -\frac{13}{9}e + \frac{15}{9} - \frac{2}{9}e + \frac{2}{9}\\
	=& -\frac{15}{9}e + \frac{17}{9}\\
	=& \frac{17}{9} - \frac{5}{3}e
\end{align*}

\end{document}









