\documentclass[11pt, a4paper]{article}

\usepackage[utf8]{inputenc}
\usepackage{fullpage}
\usepackage{graphicx}
\graphicspath{ {images/} }
\usepackage{tabularx}
\usepackage{listings}
\usepackage{mathtools}
\usepackage{amssymb}
\usepackage{eurosym}
\usepackage[ngerman]{babel}
\usepackage{cancel} %Kürzen

%TODO
%Hinweisen auf: Körperaufgabe, Statistik-Material zur Kombinatorik

\providecommand{\abs}[1]{\left\lvert#1\right\rvert}
\providecommand{\norm}[1]{\left\lVert#1\right\rVert}
\providecommand{\acos}{\mathrm{acos}}
\providecommand{\asin}{\mathrm{asin}}
\providecommand{\dx}{\ \mathrm{dx}}
\providecommand{\dt}{\ \mathrm{dt}}

\title{Klausurvorbereitung Statistik \\ Aufgaben}
\author{Yannick Schrör \and Christian Mielers}
\date{5. März 2015}

\begin{document}
\maketitle

\section{Ungleichungen}
Bestimme die Lösungen für folgende Ungleichungen
\begin{enumerate}
	\item $\frac{\abs{x+2}}{2} > \frac{x}{3} + 1$
		\begin{align*}
			\frac{\abs{x+2}}{2} > \frac{x}{3} + 1 &\Leftrightarrow \abs{x+2} > \frac{2x}{3} + 2 \\
			&\Leftrightarrow \abs{x+2} > \frac{2x+6}{3}
			\intertext{Fall 1: $x \in \left( -\infty, -2 \right)$}
			-(x+2) > \frac{2x + 6}{3} &\Leftrightarrow -x -2 > \frac{2x + 6}{3} \\
			&\Leftrightarrow -3x -6 > 2x + 6 \\
			&\Leftrightarrow -12 > 5x \\
			&\Leftrightarrow \frac{-12}{5} > x
			\intertext{Lösung im Fall 1: $\left( -\infty, -2 \right) \cap \left( -\infty, \frac{-12}{5} \right) = \left( -\infty, \frac{-12}{5} \right)$}
			\intertext{Fall 2: $x \in \left[ -2, \infty \right)$}
			(x+2) > \frac{2x + 6}{3} &\Leftrightarrow 3x + 6 > 2x + 6 \\
			&\Leftrightarrow x > 0
			\intertext{Lösung im Fall 3: $\left[ -2, \infty \right) \cap \left( 0, \infty \right) = \left( 0, \infty \right)$}
		\end{align*}
	\item $\sqrt{x^2 - 4x + 4} < \frac{\abs{x+1}}{2}$
		\begin{align*}
			\sqrt{x^2 - 4x + 4} < \frac{\abs{x+1}}{2} &\Leftrightarrow \sqrt{(x-2)^2} < \frac{\abs{x+1}}{2} \\
			&\Leftrightarrow \abs{x-2} < \frac{\abs{x+1}}{2}
			\intertext{Fall 1: $x \in \left( -\infty, -1 \right)$}
			-(x-2) < \frac{-(x+1)}{2} &\Leftrightarrow x-2 > \frac{x+1}{2} \\
			&\Leftrightarrow 2x-4 > x+1 \\
			&\Leftrightarrow x > 5
			\intertext{Lösung im Fall 1: $\left( -\infty, -1 \right) \cap \left( 5, \infty \right) = \emptyset$}
			\intertext{Fall 2: $x \in \left[ -1, 2 \right)$}
			-(x-2) < \frac{(x+1)}{2} &\Leftrightarrow -x+2 < \frac{x+1}{2} \\
			&\Leftrightarrow -2x+4 < x+1 \\
			&\Leftrightarrow 3 < 3x \\
			&\Leftrightarrow 1 < x
			\intertext{Lösung im Fall 2: $\left[ -1, 2 \right) \cap \left( 1, \infty \right) = \left( 1,2 \right)$}
			\intertext{Fall 3: $x \in \left[ 2, \infty \right)$}
			(x-2) < \frac{(x+1)}{2} &\Leftrightarrow 2x-4 < x+1 \\
			&\Leftrightarrow x < 5
			\intertext{Lösung im Fall 3: $\left[ 2, \infty \right) \cap \left( -\infty, 5 \right) = \left[ 2,5 \right)$}
		\end{align*}
		Damit ist die Gesammtlösung: $\emptyset \cup \left( 1,2 \right) \cup \left[ 2,5 \right) = \left( 1, 5 \right)$
\end{enumerate}
\section{Vollständige Induktion}
Yannick
\section{Komplexe Zahlen}
\subsection{Wurzelziehen}
Chrisbot
\section{Folgen}
Chrisbot
\section{Reihen}
Yannick
\section{Differentialrechnung}
Yannick
\section{Partialbruchzerlegung und Substitution}
Chrisbot
\section{Partielle Integration}
Yannick

\end{document}









