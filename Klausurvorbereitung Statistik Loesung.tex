\documentclass[11pt, a4paper]{article}

\usepackage[utf8]{inputenc}
\usepackage{fullpage}
\usepackage{graphicx}
\graphicspath{ {images/} }
\usepackage{tabularx}
\usepackage{listings}
\usepackage{mathtools}
\usepackage{amssymb}
\usepackage{eurosym}
\usepackage[ngerman]{babel}
\usepackage{cancel} %Kürzen
\usepackage{enumitem}

\providecommand{\abs}[1]{\left\lvert#1\right\rvert}
\providecommand{\norm}[1]{\left\lVert#1\right\rVert}
\providecommand{\acos}{\mathrm{acos}}
\providecommand{\asin}{\mathrm{asin}}
\providecommand{\dx}{\ \mathrm{dx}}
\providecommand{\dt}{\ \mathrm{dt}}

\title{Klausurvorbereitung Statistik}
\author{Yannick Schrör \and Christian Mielers}
\date{\today}

\begin{document}
\maketitle
\section{Wahrscheinlichkeiten}
Bei der Auswertung der Statistikklausuren aus den letzten 4 Jahren haben sich für folgende Themenbereiche folgende Häufigkeiten ergeben:\\

\begin{tabular}{r|l|r|r}
	ID 	&	Thema 										& Häufigkeit 	& Wahrscheinlichkeit\\ \hline
	1	& Anwendungen von Verteilungen 					& 7 			& 7/32 = 0.21875\\
	2	& Stichproben 									& 8				& 8/32 = 0.25000\\
	3	& Dichte- und Verteilungsfunktionen 			& 7 			& 7/32 = 0.21875\\
	4	& Wahrscheinlichkeitstabelle / Abhängigkeiten 	& 4 			& 4/32 = 0.12500\\
	5	& Urnenmodell / Kombinatorik 					& 4 			& 4/32 = 0.12500\\
	6	& Erwartungswert und Varianz 					& 2 			& 2/32 = 0.06250\\
\end{tabular}\\

\begin{enumerate}
\item Gehen Sie davon aus, dass Sie zu Übungszwecken zufällig eine Aufgabe aus den Altklausuren aussuchen und bearbeiten möchten. Berechnen Sie für jedes einzelne Thema die Wahrscheinlichkeit, dass Sie eine Aufgabe seines Typs auswählen.
\item Nehmen Sie an, eine Klausur bestünde aus zwei Aufgaben. Der nachfolgenden Tabelle können Sie entnehmen, mit welcher Wahrscheinlichkeit ein Thema in einer Altklausur drankam, wenn eines der anderen Thema ebenfalls drankam. Vervollständigen Sie die Tabelle.


\begin{figure}[h!]
\centering
\setlength\extrarowheight{5pt}
\begin{tabular}{r|cccccc|r}
	  		& 1 	& 2 	& 3 	& 4 	& 5 	& 6 	& $\sum$\\ \hline
	1 		& 0 	& 5/32 	& 2/32 	& 0  	& 0  	& 0 	& 7/32 \\
	2 		& 5/32	& 0		& 1/32 	& 0 	& 2/32 	& 0 	& 8/32 \\
	3 		& 2/32	& 1/32 	& 0 	& 3/32 	& 1/32 	& 0  	& 7/32 \\
	4 		& 0		& 0 	& 3/32 	& 0 	& 0  	& 1/32 	& 4/32 \\
	5 		& 0  	& 2/32 	& 1/32 	& 0 	& 0  	& 1/32 	& 4/32 \\
	6 		& 0 	& 0 	& 0  	& 1/32 	& 1/32 	& 0 	& 2/32 \\ \hline
	$\sum$ 	& 7/32 	& 8/32 	& 7/32 	& 4/32 	& 4/32 	& 2/33  & 1\\
\end{tabular}
\end{figure}

\item Ist das Auftreten der Aufgaben statistisch unabhängig? Nein, da $P(1 \land 2) \neq P(1) \cdot P(2)$
\item Bestimmen Sie folgende Wahrscheinlichkeiten:

	\begin{tabular}{ll}
	a) $P(5) = \dfrac{4}{32}$ \hspace{4cm} 				& c) $P(1|3) = \dfrac{2/32}{7/32} = \dfrac{2}{7}$ \\
	b) $P(3 \land 5) = \dfrac{1}{32}$ 					& d) $P((2 \land 5) \lor (3 \land 4)) = \dfrac{2}{32} + \dfrac{3}{32} = \dfrac{5}{32}$
	\end{tabular}

\end{enumerate}

\newpage

\section{Urnenmodell}
Chrisbot

\section{Dichte- und Verteilungsfunktion}
Yannick

\section{Anwendung von Verteilungen}
Chrisbot

\section{Stichprobem}

\end{document}









