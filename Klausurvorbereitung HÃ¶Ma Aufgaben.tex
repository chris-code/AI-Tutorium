\documentclass[11pt, a4paper]{article}

\usepackage[utf8]{inputenc}
\usepackage{fullpage}
\usepackage{graphicx}
\graphicspath{ {images/} }
\usepackage{tabularx}
\usepackage{listings}
\usepackage{enumitem}
\usepackage{mathtools}
\usepackage{amssymb}
\usepackage{eurosym}
\usepackage[ngerman]{babel}
\usepackage{cancel} %Kürzen
%\usepackage{comment}
%\usepackage{color}
%\usepackage{xcolor}
\usepackage{tikz}
%\usepackage{tkz-euclide}

\definecolor{DarkGreen}{rgb}{0,0.53,0}

%TODO
%Hinweisen auf: Körperaufgabe, Statistik-Material zur Kombinatorik

\providecommand{\abs}[1]{\left\lvert#1\right\rvert}
\providecommand{\norm}[1]{\left\lVert#1\right\rVert}
\providecommand{\acos}{\mathrm{acos}}
\providecommand{\asin}{\mathrm{asin}}
\providecommand{\dx}{\ \mathrm{dx}}
\providecommand{\dt}{\ \mathrm{dt}}
\providecommand\br[1]{\left(#1\right)}
\providecommand\ubr[2]{\underbrace{#1}_{#2}}
\providecommand\setequal{\overset{!}{=}}

\allowdisplaybreaks

\title{Klausurvorbereitung Höhere Mathematik \\ Aufgaben}
\author{Yannick Schrör \and Christian Mielers}
\date{5. März 2015}

\begin{document}
\maketitle

\section{Ungleichungen}
Bestimme die Lösungen für folgende Ungleichungen
\begin{enumerate}
	\item $\frac{\abs{x+2}}{2} > \frac{x}{3} + 1$
	\item $\sqrt{x^2 - 4x + 4} < \frac{\abs{x+1}}{2}$
	\item $\abs{x+5} > 4 + \sqrt{x^2 + 6x + 9}$
\end{enumerate}
\newpage
\section{Vollständige Induktion}
\begin{enumerate}[label=\alph*)]
\item
	Zeige mittels vollständiger Induktion, dass die folgende Gleichung für alle $n \in \mathbb{N}$ gilt:
	\begin{align*}
		\sum_{i=1}^n 4i - 1 &= 2n^2 + n
	\end{align*}
	\item
	Zeige mittels vollständiger Induktion, dass folgende Gleichung für alle $n \in \mathbb{N}$ gilt:
	\begin{align*}
		\sum_{i=1}^n i \cdot (i + 1) &= \frac{n\cdot(n+1)\cdot(n+2)}{3}
	\end{align*}
\end{enumerate}

\newpage
\section{Komplexe Zahlen}
	\begin{description}
		\item[Betrag] $\abs{x+yi} = \sqrt{x^2 + y^2}$
		\item[Polarkoordinatenform] $x+yi = r \left( \cos{\alpha} + i \sin{\alpha} \right) \text{mit } r=\abs{x+yi}, \alpha = \acos{\frac{x}{r}} = \asin{\frac{y}{r}}$
		\item[Polarkoordinaten potenzieren] $\left( r \left( \cos{\alpha} + i \sin{\alpha} \right) \right)^{p} = r^p \left( \cos \left( p \alpha \right) + i \sin \left(p \alpha \right) \right)$
		\item[Polarkoordinaten n-te Wurzel ziehen] $w_k = \sqrt[n]{r} \left( \cos \frac{\alpha + 2k\pi}{n} + i \sin \frac{\alpha + 2k\pi}{n} \right)$ \\ für $k = 0 \dots n-1$
	\end{description}
	\begin{enumerate}
		\item Bestimme den Realteil, Imaginärteil, Betrag und die komplex Konjugierte von $\frac{2+i}{1-i}$
		\item Berechne $(1-i)^{32}$
		\item Für welche $p\in\mathbb{C}$ gilt $p^4 + 4 = 0$?
	\end{enumerate}

\newpage
\section{Folgen}
\begin{description}
	\item[Folgen 1. Ordnung] $a_{n+1} = q \cdot a_n + d$
		\begin{align*}
			a_n = \begin{cases}
				a_0 \cdot q^n + d \cdot \frac{1-q^n}{1-q}, &\text{ falls } q \in \mathbb{R} \setminus \{1\} \\
				a_0 + d \cdot n &\text{ falls } q=1
					\end{cases}
		\end{align*}
	\item[Folgen 2. Ordnung] $a_{n+1} = b a_n + c a_{n-1}$ \vspace{0.3cm} \\
		Charakteristisches Polynom bilden: $q^2 = bq+c$ \vspace{0.3cm} \\
		\begin{tabular}{|c|l|}
			\hline
			Diskriminante & Explizite Darstellung \\ \hline
			$>0$ & $a_n = C_1 q_1^n + C_2 q_2^n$ \\
			$=0$ & $a_n = C_1 q_1^n + C_2 n \cdot q_2^n$ \\
			$<0$ & $a_n = C_1 r^n \cos(n\alpha) + C_2 r^n \sin(n\alpha)$ \\ \hline
		\end{tabular}
\end{description}
Finde eine explizite Darstellung für $a_n$. Gib auch den Grenzwert $\displaystyle\lim_{n \mapsto \infty} a_n$ an.
\begin{enumerate}
	\item $2 \cdot a_{n+3} = 10 \cdot a_{n+2} + 16, \qquad a_1 = 28$
	\item $a_{n+1} = 0 \cdot a_n - 4 \cdot a_{n-1}, \qquad a_0 = 1, a_1 = 3$
\end{enumerate}

\section{Reihen}
Sind die folgenden Reihen konvergent oder divergent?
\begin{enumerate}
	\item $\sum_{k=0}^\infty \frac{(k-1)(k+3)}{2(k-2)(k-4)}$
	\item $\sum_{k=0}^\infty \frac{(-1)^k 2^{k-1} - 6 \cdot 5^k}{3^{2k+1}}$
\end{enumerate}

\newpage
\section{Differentialrechnung}
Bestimme den maximalen Definitionsbereich in $\mathbb{R}$, die Nullstellen, die Grenzwerte an den Randpunkten des Definitionsbereichs, die lokalen und globalen Extrema sowie die Monotonie folgender Funktion:
\begin{align*}
	f(x) &= (x+4) \cdot (2-x^2)^{-\frac{1}{2}} - 3
\end{align*}

\newpage
\section{Partialbruchzerlegung und Substitution}
\begin{align}
	\int_a^b \mathrm{f}( \ \underbrace{\mathrm{g}(x)}_t \ ) \dx = \int_{\mathrm{g}(a)} ^{\mathrm{g(b)}} \mathrm{f}(t) \dt
\end{align}

Berechne $\int_2^3 \frac{6 \sin(x) \cos(x)}{\sin^2(x) + \sin(x) - 2 } dx$

\newpage
\section{Partielle Integration}

Löse folgendes Integral:

\newcommand{\lowerBorder}{{0}}
\newcommand{\upperBorder}{{\frac{1}{3}}}

\begin{align*}
	& \int_\lowerBorder^\upperBorder e^{3x} \cdot (2x - 5) \dx\\
\end{align*}

\end{document}









